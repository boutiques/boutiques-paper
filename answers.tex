\documentclass[a4]{article}
\usepackage{graphicx}
\usepackage{color}
\usepackage[x11names,dvipsnames,table]{xcolor} %for use in color links

\usepackage{hyperref}
\usepackage{xspace}
\newenvironment{review}%          environment name
{\textbf{Reviewer comment:}\begin{quote}}% begin code
{\end{quote}}%  
\newenvironment{answer}%          environment name
{\textbf{Answer:}\begin{quote}}% begin code
{\end{quote}}%  

\newcommand{\todo}[1]{\color{red}#1\color{black}}

\newcommand{\revised}[1]{\color{blue} #1\color{black}\xspace}

\usepackage{hyperref}
\hypersetup{colorlinks=true,linkcolor=blue,citecolor=black,filecolor=black,urlcolor=black}

\begin{document}

\section{Reviewer \#1}

\begin{review}
This is a very clear and readable manuscript describing an apparently
simple but powerful framework for constructing automated workflows
from command-line-driven applications. While reading the manuscript, I
kept thinking "this sounds very like CWL", so I was pleased to see an
extended discussion of the similarities and differences between the
two systems. In general I think it is helpful to have a few competing
solutions in domains such as this. My only, small, critique, is that
the words "command-line" should be present in (ideally) the title and
(at least) the abstract in order to qualify "application(s)".
\end{review}

\begin{answer}
We thank the reviewer for his valuable feedback. We added the words
``command-line'' in the title and abstract as per the reviewer's
suggestion.
\end{answer}

\section{Reviewer \#2}

\begin{review}
Boutiques is a specification that abstracts command-line programs,
their arguments, and their expected outputs. The goal of the
specification is to simplify application integration, reduce code
redundancy, and support reproducible science. Following the Boutiques
specification, a command-line program is documented in a descriptor
file, which follows the JSON specification. A descriptor file
specifies the arguments and expected outputs of a program. The file
includes a template of the command-line instruction, the location of a
software container that includes the program, information about each
parameter (e.g., what type of argument it expects, if any), and
relationships within groups of related parameters (e.g., parameters in
a group might be mutually exclusive). Potential arguments to a program
may be validated against the information in that program's
descriptor. Early validation is a key feature of the Boutiques
specification, because it identifies issues to the user prior to the
program's runtime. The command-line template is also advantageous
because it allows for shell functions to be used, like piping and
input/output redirection. As a side effect of the level of detail
required by the Boutiques specification, the descriptor file also
serves as precise usage documentation of a program.

Moreover, the paper describes the Boutiques core tools, which I
understood to be a Python implementation of the Boutiques
specification. These core tools include `bosh`, a command-line program
which enables users to validate Boutiques descriptors and invocation
files, easily run containerized programs, convert BIDS apps to
Boutiques descriptors, and publish their Boutiques descriptors to the
NeuroLinks website. `bosh` can run containers in Docker, Singularity,
or rootfs format. This feature satisfies the needs of both local
execution and execution on high performance computing clusters. The
inclusion of containers in the Boutiques specification lessens the
burden on the user because the user does not have to install the
software on the host machine.
\end{review}

\begin{answer}
  We thank the reviewer for his thorough review of our manuscript.
\end{answer}


\begin{review}
I would recommend a minor revision of this paper. The descriptor
schema and core tools were described and reasoned well, but the
overall goal of Boutiques could be clarified earlier in the paper. I
found the last sentence of the section titled "Other frameworks" to be
a concise and informative summary of the goal of Boutiques and the
advantages of Boutiques over other, similar frameworks. The authors
might consider adding a similar sentence to the introduction of the
paper.
\end{review}

\begin{answer}
The main features of Boutiques mentioned in section ``Other
frameworks'' are (1) its support for containerized applications and
(2) its independence from any particular programming language. While
(1) was already mentioned in the introduction, (2) indeed wasn't. We
updated the following sentence in the introduction, to address this
issue:

\begin{quote}
  Such formal descriptions, simply referred
to as \emph{descriptors}, \color{blue}can be parsed in any programming language,
can describe command lines regardless of their implementation,\color{black}
and link to a container image where the application is
installed.
\end{quote}

\end{answer}

\begin{review}
The work could also benefit from minor clarification of the descriptor
specification and minor updates to the core tools. Overall, I found
the descriptor schema well-documented, intuitive, modular, and
relatively complete (keeping in mind the ability to specify custom
parameters). I also found the various listings in the paper and the
online documentation helpful. I installed Boutiques version 0.5.3
(Python 3.6.3 on OS X 10.11.6) to test the minimal Boutiques
descriptor in Listing 5. This descriptor passed the validation step,
but when attempting to conduct a simulation run of the descriptor
given an input JSON file with the contents `{"param": 5}`, a
`KeyError` was produced, indicating that the key "container-image" was
not present in the descriptor. This led me to wonder whether
"container-image" is a required field in the descriptor schema and
whether `bosh` can execute on bare metal.
\end{review}

\begin{answer}
  The reviewer is right that no test of bare metal tools were included
  in the code base and that `bosh exec` was crashing on them. This was
  fixed in \url{https://github.com/boutiques/boutiques/pull/200}. We
  thank the reviewer for spotting this.
\end{answer}


\begin{review}
When trying to run the example in
\url{https://github.com/boutiques/boutiques/tree/5678d6f84bc1863741c6b8c595f037f622936c5c/tools/python/boutiques/schema/examples/example1/tools/python/boutiques/schema/examples/example1},
validation of the invocation and descriptor functioned as
described. For instance, validation failed when `enum\_input` and
`num\_input`, mutually exclusive parameters, were both
specified. Validation did not fail, however, when attempting to run a
'simulation' or 'launch' with the same invalid inputs. The authors
should consider including the validation step when running the program
with `bosh exec`.
\end{review}

\begin{answer}
  We thank the reviewer for spotting this. This was fixed in
  \url{https://github.com/boutiques/boutiques/pull/200}.

  We released a
  new version of Boutiques (0.5.5) that contains the fixes 
  mentioned above.
\end{answer}

\begin{review}
Furthermore, the feature of `bosh` to upload descriptors to a common repository (i.e., NeuroLinks) is an important one. While links to descriptors are certainly useful, I would like to suggest a common repository of the descriptor files themselves. Such a repository would reduce redundancy in coding effort because contributors could see which programs already have descriptors. Upstream software maintainers could maintain their Boutiques descriptors in such a repository. The burden on new users might also be reduced because they would have access to all Boutiques descriptors in a single location. If a common repository like this were in place, `bosh` might be extended to accept descriptors in a more abstract form. For example, instead of specifying a path to an existing descriptor file, a user could specify "fsl.bet.v1", and `bosh` could pull the relevant descriptor from the common repository. A similar effect might be achieved using the existing JSON file that lists the
tools on the NeuroLinks website.
\end{review}

\begin{answer}
We thank the reviewer for such a relevant comment. We deliberately
chose to not publish tools to a common repository because we believe
that they should be stored in the tool repositories so that they can
be updated and tested by the tool developers as their implementation
evolves. Instead, the NeuroLinks index provides a way to reference
descriptors without having to store them directly. The search
functionality, however, is a great one and we have discussed the following 
implementation in \texttt{bosh} over the NeuroLinks index:
\begin{enumerate}
\item Make Boutiques entries in NeuroLinks immutable:
  \begin{enumerate}
  \item modify bosh publish to commit a URL of a particular commit of the descriptor instead of the latest version.
  \item add the version number of the identifier.
  \item write guidelines to accept PRs about Boutiques descriptors in Neurolinks.
  \end{enumerate}
\item Extend \texttt{bosh} by implementing:
  \begin{enumerate}
  \item \texttt{bosh search} that greps in \url{https://raw.githubusercontent.com/brainhack101/neurolinks/master/jsons/tool.json} and returns a table containing the identifiers and (truncated) descriptions of the Boutiques tools. 
  \item \texttt{bosh get} that downloads a descriptor based on its neurolinks identifier.
  \item \texttt{bosh launch <simulate|exec> <neurolinks\_id>}  that calls \texttt{bosh get} to get the descriptor and launch the tool.
  \end{enumerate}
\end{enumerate}
\end{answer}
This will be implemented as part of issue \href{https://github.com/boutiques/boutiques/issues/34}{34}. We have updated the manuscript accordingly. The paragraph on the Publisher now reads as follows:
\begin{quote}
\emph{Publisher.} As Boutiques has primarily been adopted in the neuroinformatics
community, the publisher gets more information about the described application (such as author,
website, etc.), and adds an index to it
on
NeuroLinks\footnote{\url{https://brainhack101.github.io/neurolinks}},
a repository containing links to neuroinformatics resources and
tools. We intentionally opted for publishing descriptors to an index
such as NeuroLinks rather than to a common Boutiques repository so
that descriptors remain in the tool repositories and can be maintained
by the tool developers directly. We are currently extending
\texttt{bosh} to enable users to search for Boutiques tools in
NeuroLinks using the command-line or Python API. The publishing
functionality could be extended to new repositories for other domains,
such as Bioconductor for
bioinformatics.
\end{quote}

\end{document}
